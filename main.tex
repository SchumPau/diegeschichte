\documentclass{article}

\usepackage[utf8]{inputenc}
\usepackage[T1]{fontenc}
\usepackage[left=2.1cm,right=3.1cm,bottom=3cm,footskip=0.75cm,headsep=0.5cm]{geometry}
\usepackage[onehalfspacing]{setspace}

\title{\textbf{Die Geschichte}}
\author{\textsc{Paul Schumacher}}

\begin{document}
	\maketitle
	
	\section{Prolog}
	
	Im Schneegestöber ging Vanessa auf das blaue Haus zu. Es stand am Ende einer kleinen Straße und obwohl sie noch mehrere Meter entfernt war, konnte Vanessa die Musik vom Inneren des Hauses schon hören. Sie hatte sich für diesen besonderen Abend schick gemacht. Rosé-farbener Lippenstift, einen braunen Ledermantel mit weißen Fellstreifen und schwarze Winterstiefel boten einen sehr guten Kontrast zueinander. Ihre weiße Mütze und der Schal unterstrichen das Outfit. Als sie an dem Gartentor ankam, suchte sie nach der Klingel, doch bevor sie sie fand, wurde schon die Haustür aufgerissen. \guillemotright Hallo Vanessa\guillemotleft, rief eine vertraute Stimme aus dem Eingang. Es war ihr jüngerer Bruder David und er hatte schon einen roten Becher in der Hand. Wahrscheinlich nicht sein erster, dachte Vanessa und setzte ein Lächeln auf. Obwohl sie Geschwister waren und in derselben Stadt studierten, sahen sie sich leider immer weniger, deshalb freute sie sich bei jeder Gelegenheit.
	
	Weitere Freunde von ihr gesellten sich zu David und so beeilte sie sich, ins warme Haus zu kommen. Nachdem sie ihren Bruder umarmt hat, wandte sie sich den anderen zu, die in einem Kreis um sie warteten. Sie entdeckte ihre beste Freundin Marie, die sie seit Weihnachten nicht gesehen hatte. Die meisten anderen kannte sie aber noch nicht oder nur vom Sehen in der Uni. \guillemotright Komm endlich, die Party ist schon im vollen Gang,\guillemotleft drängte sie ihr Bruder. Ein wenig aufgeregt zog sie ihre Mütze, Schal und den Mantel aus. Unter ihnen kam ein rosa Kleid hervor, welches das Dekolleté unter einem Saum von Spitze verbarg. Karl, der Gastgeber der Party, kam auf sie zu. \guillemotright Naa, wie geht’s Dir?\guillemotleft, begrüßte er sie mit einer überschwänglichen Umarmung. \guillemotright Super und Dir?\guillemotleft \guillemotright Es scheinen immer neue Gäste zu kommen, ich bin mir nicht mal sicher, ob ich sie alle eingeladen habe.\guillemotleft \guillemotright Das Haus ist doch groß, da passen bestimmt alle rein\guillemotleft, beruhigte Marie Karl. \guillemotright Bestimmt und genügend Alkohol haben wir sowieso da!\guillemotleft, lachte er.
	
	Marie kannte Karl schon seit einigen Jahren aus der Jungen Gemeinde. Neben den wöchentlichen Treffen hatte sich ein gleichaltriger Kreis gebildet, der des Öfteren etwas zusammen unternahm oder wie in diesem Fall eine Party schmiss. Aber gerade zu so einem Anlass, wie Silvester, kamen oft auch andere Freunde hinzu und so wuchs die Zahl schnell auf über fünfzig an. Ihr Bruder brachte ihr gerade etwas zu trinken, als ihr blick auf einen gutaussehenden Jungen fiel, der an einer Wand lehnte und offensichtlich mit zwei Mädchen flirtete. \guillemotright Wer ist das\guillemotleft, fragte sie ihren Bruder.
	
	\begin{center}
		$\ast$
	\end{center}

	Es ist schön so viele Freunde zu haben, die bei mir gemütlich Silvester feiern. Zwar sind es noch über zwei Stunden bis Mitternacht, aber ich kann ja mal nach dem Feuerwerk so langsam schauen. Karl ging auf die Kellertreppe zu, als ihn von hinten jemand umarmte. Es war seine beste Freundin Maren und er konnte, bevor er sich umdrehte, schon riechen, dass sie sich nicht dem Alkohol gegenüber verschlossen hatte. \guillemotright Da hast Du ja echt was auf die Beine gestellt.\guillemotleft, lobte sie ihn. \guillemotright Naja, ich habe es ja nicht allein gemacht. Zum Beispiel hast Du mir ja bei der Getränkeauswahl sehr geholfen\guillemotleft, sagte er und zwinkerte sie an. \guillemotright Das war doch kein Ding\guillemotleft, wehrte sie ab, \guillemotright mach ich doch gerne, vor allem, weil ich weiß, dass Du keine Ahnung davon hast.\guillemotleft Beide fingen sie an zu lachen. Sie kannten sich schon seit der Grundschule und hatten sehr wenig Geheimnisse voreinander. Manchmal hatte Karl das Gefühl, dass er mehr über Marens Liebesleben wusste, als ihm lieb war. Er selbst hatte zwei Freundinnen bisher gehabt. Und das, obwohl Maren mir immer wieder versichert, dass ich gut aussehe. Naja, zum Glück gibt es ja auch andere Hobbys. \guillemotright Wolltest Du etwa ein wenig Bier holen, denn das hier oben geht sehr schnell zur Neige\guillemotleft, riss Maren ihn aus seinen Gedanken. \guillemotright Ja, mach ich\guillemotleft, versicherte er ihr, noch etwas in Gedanken versunken. \guillemotright Bis gleich\guillemotleft, flötete Maren und schwebte wieder auf die Party zu. Karl ging die Treppe in den Keller herunter und hörte schon ab der Hälfte verdächtige Geräusche. Nur gut, dass ich die Schlafzimmer oben abgeschlossen habe. Wer das hier wohl sein wird? Als er um eine Ecke bog, sah er ein knutschendes Pärchen an der Wand. Das Mädchen müsste Jana sein, folglich ist der Junge Georg. Die beiden sind doch erst kurz vor Weihnachten zusammengekommen, dachte Karl amüsiert. Georg hatte sein Gesicht zwischen Janas Brüsten vergraben. Jana scheint es sehr zu gefallen, als sie ein weiteres Mal leise aufstöhnt. Doch bevor er sehen konnte, wo Jana ihre Hände hatte, riss sie erschrocken die Augen auf. \guillemotright Karl, ich habe Dich nicht kommen hören\guillemotleft, rief sie peinlich berührt. Auch Georg sah jetzt auf. \guillemotright Macht ruhig weiter, ich wollte nur etwas Bier holen. Es ist schon wieder alle\guillemotleft, beruhigte Karl die beiden. Nun konnten sie auch ein wenig lächeln und Karl wandte sich ab, um zügig einen Kasten Bier zu suchen. Er wollte die beiden möglichst schnell wieder allein lassen. Ach, hätte ich doch auch jemanden, mit dem ich so knutschen kann, dachte er betrübt.
	
	Als Karl die Treppe mit dem Kasten wieder hochgestiegen kam, erwartete ihn Maren auch schon wieder. \guillemotright Du siehst sehr verwirrt aus, hast Du einen Geist gesehen?\guillemotleft, wunderte sie sich. \guillemotright Nein, nur Jana und Georg beim rummachen\guillemotleft, erwiderte Karl. \guillemotright Grausige Vorstellung, aber komm jetzt, ich möchte Dir jemanden zeigen, den ich extra für Dich eingeladen habe.\guillemotleft Und während Maren das sagte, zwinkerte sie Karl verschwörerisch zu. Gespannt folgte er Maren durch das Getümmel der Partygäste.
	
	\begin{center}
		$\ast$
	\end{center}

	\guillemotright Das ist Martin. Er studiert auch hier in Dresden. Und wohnt mit meiner Freundin zusammen in einer WG.\guillemotleft \guillemotright Wer ist eigentlich Deine Freundin? Du hast sie mir nie vorgestellt...\guillemotleft  Sie lächelte ihren Bruder an. \guillemotright Das habe ich aus gutem Grund getan.\guillemotleft \guillemotright Ist sie Dir denn peinlich?\guillemotleft, provozierte sie ihn. \guillemotright Nein, sie ist viel zu heiß für mich\guillemotleft, ein glückliches Lächeln umspielte Davids Lippen. Vanessa schaute David an: \guillemotright Machen wir doch ein Ratespiel, ist sie hier im Raum?\guillemotleft \guillemotright Du bist doof, aber gut. Ja ist sie.\guillemotleft  Na da kann ich ja schon einige ausschließen. \guillemotright Hat sie denn ein Kleid an?\guillemotleft \guillemotright Das wird zu einfach, wenn ich Dir das beantworte\guillemotleft, sagte David. \guillemotright Also nein\guillemotleft, stellte Vanessa fest. Es gibt hier nur wenige in Davids Alter ohne Kleid. Es könnte sie mit der weißen Hose da drüben sein. Ihr Blick fiel auf ein etwa 1,60 cm großes Mädchen mit einer enganliegenden weißen Hose, die verträumt aus dem Fenster sah. Sie hatte ein blasses, fast elfenbeinfarbenes Gesicht, welches in einem perfekten Kontrast zu ihrem braunen Haar und ihren dunklen Augen stand. Weiterhin trug sie einen weiten bordeauxfarbenen Pullover, der vorne in ihre Hose gesteckt war. Diese blöde neue Mode. Es gibt viele, denen es nicht steht, aber sie gehört definitiv nicht dazu. David hatte eindeutig Recht. Sie ist heiß! \guillemotright Pass auf, ich spreche jetzt zufällig ein Mädchen an, von dem ich denke, dass es deine Freundin sein könnte, mal sehen was passiert.\guillemotleft, zog sie ihren Bruder auf. Ihr Bruder schaute sie erschrocken an: \guillemotright Das machst Du bitte nicht!\guillemotleft, rief er. \guillemotright Oh doch\guillemotleft, sagte Vanessa, während sie sich auf das Mädchen zubewegte. Dann kann sie mich ja vielleicht ihrem Mitbewohner vorstellen, dachte Vanessa. David lief ihr hinterher. Wahrscheinlich will er das schlimmste verhindern. Ich mache es nicht zu peinlich. \guillemotright Hey Du, wie heißt Du denn\guillemotleft, ging sie auf das Mädchen zu. Dieses schaute sich schüchtern um. Offenbar kennt sie hier keinen auf der Party. \guillemotright Sophie\guillemotleft, erwiderte sie leise. Sehr schüchtern. \guillemotright Und Du kennst meinen Bruder besser\guillemotleft, fragte Vanessa sie. \guillemotright Nun ja..., wir sind seit einem viertel Jahr zusammen\guillemotleft, antwortete Sophie. \guillemotright So lange schon?\guillemotleft  Vanessa drehte sich zu ihrem Bruder um. \guillemotright Das wird noch ein Nachspiel haben, Brüderchen. So lange verheimlichst Du mir keine Freundin.\guillemotleft \guillemotright Hast Du denn einen Freund\guillemotleft, ging Vanessa in die Offensive, bevor David antworten konnte. \guillemotright Nein, aber ich habe gehofft, dass Du mich jemandem vorstellen kannst\guillemotleft, erwiderte Vanessa. \guillemotright Aber ich kenne hier doch niemanden außer David und Martin\guillemotleft, entschuldigte sich Sophie. \guillemotright Es geht aber genau um Martin. Wie lange wohnt ihr denn schon in einer WG?\guillemotleft \guillemotright So um die sieben Monate. Aber wenn Du das unbedingt möchtest, kann ich ihn Dir vorstellen\guillemotleft, ließ sich Sophie breitschlagen. Nachdem Sophie David einen Kuss auf die Lippen gehaucht hatte, gingen die beiden Mädchen auf Martin zu, der mittlerweile allein an einem Tisch stand.
\end{document}